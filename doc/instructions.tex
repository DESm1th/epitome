%% Harford User Manual, Viviano, Joseph David Apr. 2014
\documentclass[final,titlepage,letterpaper,oneside,12pt]{article}
\usepackage[usenames,dvipsnames]{xcolor} % colors
\usepackage{tikz}                       % graphs
\usepackage{caption}                    % figure captions
\usepackage[utf8]{inputenc}             % utf8 encoding compatibility
\usepackage{fourier}                    % fonts
\usepackage[T1]{fontenc}                % fonts
\usepackage{natbib}                     % bibliography
\usepackage[colorlinks=true, pdfstartview=FitV, linkcolor=MidnightBlue, citecolor=MidnightBlue,urlcolor=MidnightBlue]{hyperref}                % custom colors
\usetikzlibrary{arrows}                 % arrows
\usepackage{amsmath}                    % equations
\usepackage{epigraph}                   % quotation
\renewcommand{\texttt}[2][BrickRed]{\textcolor{#1}{\ttfamily #2}}% \texttt[<color>]{<stuff>}
\newcommand{\atilde}{\raise.17ex\hbox{$\scriptstyle\mathtt{\sim}$}}
\renewcommand\epigraphflush{flushright} % 
\renewcommand\epigraphsize{\normalsize} % 
\setlength\epigraphwidth{0.7\textwidth} % 
\definecolor{titlepagecolor}{cmyk}{1,.60,0,.40} % custom color

\newenvironment{blockquote}{%
  \par%
  \medskip
  \leftskip=4em\rightskip=2em%
  \noindent\ignorespaces}{%
  \par\medskip}

%% Title Page
\makeatletter                       
\def\printauthor{%                  
    {\normalsize \@author}}              
\makeatother
\author{%
    Joseph D. Viviano \\
    Department of Biology \\
    York University \\
    Toronto, ON, CA \\
    joseph.d.viviano@gmail.com
}

\newcommand\titlepagedecoration{%
\begin{tikzpicture}[remember picture,overlay,shorten >= -10pt]

\coordinate (aux1) at ([yshift=-15pt]current page.north east);
\coordinate (aux2) at ([yshift=-410pt]current page.north east);
\coordinate (aux3) at ([xshift=-4.5cm]current page.north east);
\coordinate (aux4) at ([yshift=-150pt]current page.north east);

\end{tikzpicture}%
}

\begin{document}
\begin{titlepage}

\noindent
\begin{center}
\begin{Huge}
EPItome-xl\par
\begin{small}
on\par
\end{small}
Harford\par
\end{Huge}
\end{center}
\vspace*{2cm}
\epigraph{As long as our brain is a mystery, the universe, the reflection of the structure of the brain will also be a mystery.}%
{\textsc{Santiago Ramón y Cajal}}
\null\vfill
\vspace*{1cm}
\noindent
\hfill
\begin{minipage}{0.55\linewidth}
    \begin{flushright}
        \printauthor
    \end{flushright}
\end{minipage}
%
\begin{minipage}{0.02\linewidth}
    \rule{1pt}{72pt}
\end{minipage}
\titlepagedecoration
\end{titlepage}

% Table of contents
\tableofcontents
\newpage

%%%%%%%%%%%%%%%%%%%%%%%%%%%%%%%%%%%%%%%%%%%%%%%%%%%%%%%%%%%%%%%%%%%%%%%%%%%%%%%
% Introduction
%%%%%%%%%%%%%%%%%%%%%%%%%%%%%%%%%%%%%%%%%%%%%%%%%%%%%%%%%%%%%%%%%%%%%%%%%%%%%%%
\section{Introduction}

This server was built by a complete stranger to server-building, and I undoubtedly made some silly mistakes along the way. This document is my effort to crystallize these decisions for those who paid for it (henceforth \textit{the money}) and whoever takes over responsibility from me (henceforth \textit{the unlucky}). Hopefully this isn't my last job ever, and I still walk this earth. If that is true, feel free to contact me at any time \texttt{joseph.d.viviano@gmail.com} 

%%%%%%%%%%%%%%%%%%%%%%%%%%%%%%%%%%%%%%%%%%%%%%%%%%%%%%%%%%%%%%%%%%%%%%%%%%%%%%%
% How to Connect to the Server Remotely
%%%%%%%%%%%%%%%%%%%%%%%%%%%%%%%%%%%%%%%%%%%%%%%%%%%%%%%%%%%%%%%%%%%%%%%%%%%%%%%
\section{Connecting}

To facilitate easy access to the server from your personal computer, on or off campus, we have configured three standard access methods.

\subsection{Samba: Access to Shared Files On Your Desktop}

All non-MRI data is accessible and editable through the finder/explorer window right on your desktop through a file-sharing service called Samba, assuming you have permission to do so. Contact an administrator to change permissions.

This is a good system for accessing behavioural data stored on the server. Some of this data is protected, and may not be accessible unless you are a member of the \texttt{cnnlabdata} group. An administrator can add you to this group if you need access.

\subsubsection{Mac OSX}

First, click on your desktop. In your menu bar, select \texttt{go}, \texttt{connect to server}, and type \texttt{smb://130.63.40.171}, after which you need only to enter your username and password for the server.

\subsubsection{Windows}

First, click on your start menu, and open \texttt{run}. Type \texttt{\\130.63.40.171}, hit enter, and enter your username / password for the server. You should now have access to the CNNLAB folders in your Windows Explorer program.
 
\subsubsection{Administering Samba}

Samba is annoying and maintains its own set of passwords and permissions. This is particularly annoying because you will need to find an elegant way for users to set this password up. I haven't.

First, you need to add an existing user to samba using \texttt{smbpasswd -a [
username]}. Next we add \texttt{[username]}  to the \texttt{/etc/samba/smbusers} file in the form \texttt{[username] = "[username]"}. I've configured samba to mount \texttt{/srv/CNNLAB} as a writable directory structure in \texttt{/etc/samba/smb.conf} under the \texttt{[CNNLAB]} section. More directories could be added in this way. 

\subsection{SSH}

If you are comfortable with terminal-only access, you can use the \texttt{ssh} command on all Mac or Linux systems (normally, this is already installed). Windows users should install \href{http://www.chiark.greenend.org.uk/~sgtatham/putty/}{PuTTY} to gain access to the \texttt{ssh} command in the Windows terminal.

The major advantage of terminal-only access is lag-free operation on very slow network connections, and the ability to log in to the server from any computer with zero configuration. To access the server, enter the command \texttt{ssh -p 45100 [your username]@130.63.40.171}, where \texttt{[your username]} is the login name registered on Harford. If you require a username, please contact one of the administrators. You will next need to enter your case-sensitive password associated with your user account and should have access to the server.

\subsection{x2go}

Normally, you will want to view a full graphic-user interface while you interact with the server. This will allow you to easily view the data remotely, and make use of various programs not available with terminal-only access. To do this, we use x2go (\href{http://wiki.x2go.org/doku.php}{download x2go client here}). On Windows and Mac, you should be able to open x2go by clicking on the installed icon. On Linux, you may need to type \texttt{x2goclient} into the terminal.

Configuring x2go is fairly straightforward. First, click the `new session' button, and set these values:

\begin{itemize} \itemsep-2pt
    \item{Session name: [whatever you want ... I chose Harford]}
    \item{Host: 130.63.40.171}
    \item{Login: [your username]}
    \item{SSH port: 45100}
    \item{Session type: XFCE}
\end{itemize}

You might notice that this is almost identical to the information that we use to \texttt{ssh} into the server, because that is what we are actually doing! The only additional setting regards the \texttt{session type}, as we now need to specify the graphic user interface.

The rest of the items can remain blank or at their defaults. You do not need to configure anything specifically under the connection or misc tab. However, here you can set performance values to your liking, especially to compensate for a slow connection. Try moving the slider to the left before changing the picture quality settings manually. Finally, click `OK' and log into the server by clicking on the newly created button on the right hand side. You will need to enter your password for the server here.

\subsubsection{Troubleshooting x2go}

In some cases, you will receive an error involving the RSA key, and will not be allowed to log into the server. Your RSA keys are stored in \texttt{\atilde/.ssh/known\_hosts}. You can inspect the contents of this file by typing \texttt{cat \atilde/.ssh/known\_hosts} into the terminal. If you find an entry that starts with \texttt{130.63.40.171}, remove it using either: \texttt{gedit \atilde/.ssh/known\_hosts} or \texttt{vi \atilde/.ssh/known\_hosts} (\href{http://glaciated.org/vi/}{here are some instructions for vi}). If you are having trouble doing this, you can simply reset the \texttt{known\_hosts} file by removing it with \texttt{rm \atilde/.ssh/known\_hosts} (this might affect other systems relying on RSA keys, but I will assume it won't if you are following these instructions). You should now be able to log into the server using x2go.

By default, XFCE via x2go does not play well with OSX, a common symptom being the \texttt{tab} and \texttt{p} keys not working properly. I have configured the server to play nicely with Macintosh computing devices, but in case this needs to be resolved again, you simply need to go to your \texttt{~/.config/xfce4/xfconf/xfce-perchannel-xml/} directory and delete the \texttt{keyboard-layouts.xml} and \texttt{xfce4-keyboard-shortcuts.xml} files, log out, and log back in again.

\subsection{Passwords}

Your default password can be changed to something more easily remembered by doing the following:

\begin{enumerate} \itemsep-2pt

    \item{SSH into the server using port \href{http://en.wikipedia.org/wiki/Looking_Glass_Studios}{45100}.}

    \begin{itemize} \itemsep-2pt
        
        \item{Harford uses a non-standard SSH port to prevent random attacks. This requires you to specify it at the command line with the following command: \texttt{ssh -p 45100 [your username]\@130.63.40.171}.}

    \end{itemize}
    
    \item{Use the \texttt{passwd} command.}
    
    \begin{itemize} \itemsep-2pt
        
        \item{If you type passwd at the command line, you will be prompted for your old annoying password, and asked to enter a new one.Your new password must satisfy the following conditions:}
        
        \begin{itemize} \itemsep-2pt
            \item{1 Uppercase letter}
            \item{1 lowercase letter}
            \item{1 number}
            \item{1 symbol (i.e., *, \_\#\$\%)}
            \item{12 characters length minimum.}
        \end{itemize}
        
        \item{This does not need to be hard to remember! Computer no think like human.}
        
        \begin{itemize} \itemsep-2pt
            
            \item{\texttt{\#sQ\_} is easy for a computer to guess.}

            \item{\texttt{Br4in-fun-4eva!} is much harder.}

        \end{itemize}
    \end{itemize}
\end{enumerate}

Be sure to write your password down. If you forget your password, an administrator can reset it for you. If you \textit{are} an administrator, you should know you can change \texttt{username}'s password by dropping to \texttt{root} and typing \texttt{passwd [username]}.

Also, recall that samba and unix passwords are \textit{not} automatically synchronized. [This really should be fixed.]

%%%%%%%%%%%%%%%%%%%%%%%%%%%%%%%%%%%%%%%%%%%%%%%%%%%%%%%%%%%%%%%%%%%%%%%%%%%%%%%
% The Hardware we have here
%%%%%%%%%%%%%%%%%%%%%%%%%%%%%%%%%%%%%%%%%%%%%%%%%%%%%%%%%%%%%%%%%%%%%%%%%%%%%%%
\section{Hardware}

Harford is a \texttt{Thinkserver TS440 ($8 \times 3.5"$) HDD Hot-Swappable} server loaded with 4 HHD at the moment. Hard drive caddy part \texttt{\# 03X3969, FRU HS 3.5” HDD Tray V3.0}. It is loaded with 4 $\times$ Western Digital RED 4.0 TB 5400 HDD, configured in RAID 1+0. This currently leaves us with 8 TB of usable hard drive space, striped arrays, and on-site redundancy. It can easily be expanded to 2$\times$ that with an additional 4 drives.

\subsection{Hardware List}

Currently, Harford contains:

\begin{enumerate} \itemsep-2pt
    \item{Thinkserver TS440 [70AQ000CUX].}
    \item{20 GB RAM [2$\times$8 GB added, 2$\times$2 GB stock].}
    \item{16 TB of WD RED drives in RAID 1+0 for 8 TB of usable space.}
    \item{NVIDIA GeForce 640 1GB RAM GPU.}
    \item{Intel Xeon E3-1245V3 / 3.4 GHz $\times$ 8 cores.}
\end{enumerate}

%%%%%%%%%%%%%%%%%%%%%%%%%%%%%%%%%%%%%%%%%%%%%%%%%%%%%%%%%%%%%%%%%%%%%%%%%%%%%%%
% The way I set things up
%%%%%%%%%%%%%%%%%%%%%%%%%%%%%%%%%%%%%%%%%%%%%%%%%%%%%%%%%%%%%%%%%%%%%%%%%%%%%%%
\section{Server Architecture}
\subsection{Partitions}

At the moment, Harford consists of three partitions: 

\begin{enumerate}
    \item{\texttt{/boot}: a 200 MB partition that is integral to the life of the server. This contains the \textit{Linux kernels} of the system -- the server's brainstem. Do not mess with this partition.}
    
    \item{\texttt{/}: a 75 GB partition containing all user \texttt{/home} folders, software, and the operating system. This \textit{should} be large enough for indefinite expansion, if people don't store data in their home folders.}
    
    \item{\texttt{/srv}: a 6.59 TB partition containing all of our data.}
\end{enumerate}


\begin{center}
\textbf{NB: this means that the lion's share of the disk space is found under \texttt{/srv} and therefore large files should \textit{always} be kept there!}
\end{center}

\subsection{Operating System}

I chose Ubuntu Server 12.04 LTS for its excellent support, modern features, and compatibility with the NeuroDebian project (\url{neuro.debian.net}). I named the server Harford after the Kubrickian hero. When I first installed the server, I was left only with a basic terminal. To get things normal-looking, I had to \texttt{sudo apt-get install} the following:

\noindent
\begin{small}
\begin{itemize} \itemsep-2pt
    \item{openssh-server}
    \item{xfce4}
    \item{kde-plasma-desktop}
    \item{synaptic}
    \item{lightdm-gtk-greeter}
    \item{jockey-gtk}
    \item{dmz-cursor-theme} 
    \item{xubuntu-icon-theme}
    \item{elementary-icon-theme}
\end{itemize}
\end{small}


This will produce a basic desktop environment for you to work in. I include both the \texttt{xfce4} and \texttt{kde} desktop environments, which each have their own merits and utility. They essentially control the graphic user interface of the server but do not differ substantially in actual usability.

A note on what these basic programs do. \texttt{openssh-server} enables SSH access through the terminal and is how people can control the server remotely. \texttt{xfce4} and \texttt{kde-plasma-desktop} are two alternative graphic user interfaces for the computer. \texttt{synaptic} allows one to probe the Internet for possible software installations and largely automates that process. Most of the software on the server can be installed and/or uninstalled using this simple program. \texttt{lightdm-gtk-greeter} presents the user with a login screen. The remaining installs are all graphic user interface niceties.

\subsection{Directories}

The Unix file system is hierarchical: the top of this tree is at \texttt{/}. As mentioned previously, the bulk of the data is mounted on a separate partition under \texttt{/srv/}. Some optional software that wasn't installed by using \texttt{apt-get} / \texttt{synaptic} resides in \texttt{/opt/}. Under\texttt{/home/} resides each user's personal folder containing their desktop, personal files, and personal settings (such as the \texttt{\atilde/.bashrc} file). Everything else is pretty much standard.

Under \texttt{/srv/} reside the following major directories:
\begin{itemize} \itemsep-2pt
    \item{CNNLAB: Contains shared documents and behavioural data.}
    \item{CODE: Contains the pipeline \& analysis code.}
    \item{FTP: A web-accessible FTP for data-sharing across the globe. Currently not password protected.}
    \item{LOGS: Log files generated by \texttt{CRON} jobs.}
    \item{MOVER: Files that only exist to be sorted through after import. I hope this folder isn't permanent.}
    \item{MRI: All the MRI (and possibly MEG) data.}
\end{itemize}

In \texttt{MRI}, four directories exist: \texttt{ANALYSIS}, \texttt{QUARANTINE}, \texttt{RAW}, and \texttt{WORKING}. Under each of these folders is a mirrored set of experimental name directories. Each experiment is given a short name (e.g., `TRSE', `BEBASD', `COP'). \texttt{ANALYSIS} is a convenience folder for those who would like to use the server's resources to analyze their experimental data. \texttt{QUARANTINE} is a safe place for data one would like to remove from the \texttt{WORKING} directory, but not delete. Generally `bad runs' should be moved here instead of being deleted. \texttt{WORKING} is where the NIFTI-converted data resides and is accessed by the pipeline. \texttt{RAW} contains exact copies of the raw DICOM data, however it was found.

Under each experiment directory, the following structure is found:

\begin{itemize} \itemsep-2pt
    \item{Subject IDs. Experiment-wide files are placed here.}
    \item{Subject file types (e.g., REST, T1, TASK). Names are arbitrary but must be consistent across subjects.}
    \item{File type session folders. Any across-session data also ends up here.}
    \item{Run folders. Each run folder should contain exactly one NIFTI file. In some cases it is appropriate to place run-specific data here (e.g., physiological recordings). Intermediate stages of pre-processing are also stored here.}
\end{itemize}

\subsection{Permissions}

The MRI data is separated into 4 main branches: RAW, WORKING, QUARANTINE, and ANALYSIS. These different tiers are more or less editable by various users of the server to strike a balance between data-security and usability. As a general rule, we also don't let unauthorized people look at data they aren't supposed to for both privacy and competitive reasons. These permissions are maintained by a \texttt{root}-owned nightly \texttt{CRON} job {\color{red}\texttt{maintain\_permissions.sh}}.

The \texttt{grandvizier} user should be used by the system administrators to preform various tasks without the need of dropping to \texttt{root}, which I am trying to discourage as it can be dangerous to spend too much time with so much power. If you can see a file owned by the \texttt{grandvizier}, that typically means it is being protected.  

\subsubsection{RAW}

These are DICOM files. Mostly used for archival purposes. Shouldn't be edited.

\begin{flushleft}
\texttt{owner = grandvizier, rwx \\
        group = staff      , r-x \\
        else  = ---} \\
\end{flushleft}

\subsubsection{WORKING}

These are the files manipulated by the pipeline code. Generally, these files should only be accessed by the pipeline and not manually. Right now, experiment specific group-wise permissions allow for you to go in and delete everything \textit{except} the input RAW data at the bottom of the WORKING tree. This allows you to `reset' problem subjects or whole users using the {\color{red}\texttt{cleanup\_X.sh}} programs, or manually remove problem files.

\begin{flushleft}

\texttt{owner = grandvizier, rwx \\
        group = [experiment], rwx (except inputs which are r-x) \\
        else  =            , ---} \\
\end{flushleft}

\subsubsection{ANALYSIS}

These folders are where people \textit{are} allowed to mess around. Generally, outputs from the pipeline can be copied into the ANALYSIS tree for manipulation. This is zero-risk as there is always an identical copy of the pre-processed files in the WORKING directory.

\begin{flushleft}
\texttt{owner = grandvizier, rwx \\
        group = [exrperiment], rwx \\
        else  =            , ---} \\
\end{flushleft}

\subsubsection{Privacy Notes}

The \texttt{grandvizier} user is special, and shared among \textit{the money} and \textit{the unlucky}. The \texttt{grandvizier} is also capable of  destroying millions of tax-payer dollars in a one-line command, so it shouldn't be used by anyone unless required.

All pipeline code will be run by the individuals within an experiment group -- and the code will work so long as the individual is permitted to interact with a given data set. This is safe because the pipeline code isn't editable by the users in the first place, so we can't do undue harm to our data by mistake.

\subsection{Security}
\subsection{Backup}

%%%%%%%%%%%%%%%%%%%%%%%%%%%%%%%%%%%%%%%%%%%%%%%%%%%%%%%%%%%%%%%%%%%%%%%%%%%%%%%
% The stuff I installed
%%%%%%%%%%%%%%%%%%%%%%%%%%%%%%%%%%%%%%%%%%%%%%%%%%%%%%%%%%%%%%%%%%%%%%%%%%%%%%%
\section{Software \& Configuration}

The following is a list of the software installed on the server, and if you are lucky, it is even up-to-date.

\noindent
\begin{small}
\begin{itemize} \itemsep-2pt
    
    \item{\href{http://neuro.debian.net/}{NeuroDebian}:}
    \begin{itemize} \itemsep-2pt
        \item{\href{http://fsl.fmrib.ox.ac.uk/fsl/fslwiki/}{FSL 5.0.6}}
        \item{\href{http://www.mccauslandcenter.sc.edu/mricro/mricron/dcm2nii.html}{DICOM2NIFTI}}
        \item{\href{http://afni.nimh.nih.gov/afni/}{AFNI}}
        \item{\href{https://surfer.nmr.mgh.harvard.edu/}{FreeSurfer}}
        \item{\href{http://brainvis.wustl.edu/wiki/index.php/Caret:About}{Caret}}
        \item{\href{http://nipy.org/nibabel/}{Python: NiBabel}}
    \end{itemize}
    
    \item{General Computing:}
    \begin{itemize} \itemsep-2pt
        \item{\href{http://cran.r-project.org/}{R}}
        \item{\href{https://wiki.gnome.org/Apps/Gedit}{gedit}}
        \item{\href{https://launchpad.net/terminator}{Terminator}}
        \item{\href{https://www.libreoffice.org/}{LibreOffice}}
        \item{\href{https://wiki.gnome.org/Apps/Evince}{Evince}}
        \item{\href{https://wiki.gnome.org/Apps/Evince}{Firefox}}
        \item{\href{http://www.inkscape.org/en/}{Inkscape}}
        \item{\href{http://www.gimp.org/}{Gimp}}
        \item{\href{https://wiki.gnome.org/Apps/Dia}{Dia}}
        \item{\href{http://www.videolan.org/vlc/index.html}{VLC}}
        \item{\href{https://www.samba.org/}{Samba}}
        \item{\href{http://www.socher.org/index.php/Main/HowToInstallSunGridEngineOnUbuntu}{Oracle Sun-Grid Engine}}
    \end{itemize}
    
    \item{Python Packages:}
    \begin{itemize} \itemsep-2pt
        \item{\href{http://www.numpy.org/}{NumPy}}
        \item{\href{http://www.scipy.org/}{SciPy}} 
        \item{\href{http://pandas.pydata.org/}{Pandas}}
        \item{\href{http://matplotlib.org/}{matplotlib}}
        \item{\href{http://scikit-learn.org/stable/}{scikit-learn}}
        \item{\href{http://scikit-image.org/}{scikit-image}}
        \item{\href{http://networkx.github.io/}{NetworkX}}
        \item{\href{http://www.pymvpa.org/}{PyMVPA}}
        \item{\href{http://ipython.org/}{iPython}}
    \end{itemize}

    \item{Misc -- Installed Manually in \texttt{/opt}:}
    \begin{itemize} \itemsep-2pt
        \item{\href{https://gephi.org/}{Gephi}}
        \item{\href{http://www.mathworks.com/products/compiler/mcr/}{MATLAB Compiler Runtime}}
        \item{\href{http://afni.nimh.nih.gov/sscc/dglen/McRetroTS}{McRetro}}
    \end{itemize}

\end{itemize}
\end{small}

\subsection{Oracle Sun-Grid Engine}
The server, and EPItome-xl, rely on Oracle's Sun-Grid engine to schedule jobs. At the moment this feature is merely cosmetic, but it will help immensely when multiple users are attemping to run their experiments through the pipeline at the same time (such as 4 days before SfN abstracts are due). The EPItome-xl pipeline actually handles most of the nitty-gritty here, which I will document later (i.e., {\color{red}never}). In order to install and configure the scheduler, I followed \href{http://www.socher.org/index.php/Main/HowToInstallSunGridEngineOnUbuntu}{these instructions}.  
\subsection{MRI Tools}
\subsection{Programming Languages}

%%%%%%%%%%%%%%%%%%%%%%%%%%%%%%%%%%%%%%%%%%%%%%%%%%%%%%%%%%%%%%%%%%%%%%%%%%%%%%%
% The functionality of the pipeline
%%%%%%%%%%%%%%%%%%%%%%%%%%%%%%%%%%%%%%%%%%%%%%%%%%%%%%%%%%%%%%%%%%%%%%%%%%%%%%%
\section{The Pipeline}

For the pipeline to work, your RAW data must be converted to the appropriate format in the WORKING directory. EPItome-xl should be installed and configured properly on your server (code \& up-to-date installation instructions are available from \href{https://github.com/josephdviviano/EPItome-xl}{GitHub}).

What follows is a description of what each module in EPItome-xl does. These modules can be easily chained together manually, or by using the command-line interface included with the pipeline. New modules are simply bash scripts which call various programs, including FSL, Freesurfer, AFNI, and custom python programs. Therefore, functionality of EPItome is easily extended without perturbing the function of older modules.

The types of modules included can be roughly split into 4 categories: freesurfer, pre-processing, quality-control, and cleanup.

%%%%%%%%%%%%%%%%%%%%%%%%%%%%%%%%%%%%%%%%%%%%%%%%%%%%%%%%%%%%%%%%%%%%%%%%%%%%%%%
%%%%%%%%%%%%%%%%%%%%%%%%%%%%%%%%%%%%%%%%%%%%%%%%%%%%%%%%%%%%%%%%%%%%%%%%%%%%%%%
\subsection{Freesurfer}

Right now, the default freesurfer recon-all is run on every participant before further processing. This is to produce surface files that can be used for cortical smoothing / data visualization, and the automatic generation of tissue masks which can be used for the generation of nuisance regressors. 

%%%%%%%%%%%%%%%%%%%%%%%%%%%%%%%%%%%%%%%%%%%%%%%%%%%%%%%%%%%%%%%%%%%%%%%%%%%%%%%
\subsubsection{fsrecon.py}
Usage: \texttt{fsrecon.py <data\_directory> <experiment> <modality> <cores>} \

\begin{blockquote}
data\_directory -- full path to your MRI/WORKING directory. \\
experiment -- name of the experiment being analyzed. \\
modality -- image modality to import (normally T1). \\
cores -- number of cores to dedicate (one core per run). \
\end{blockquote}

\noindent This sends each subject's T1s through the Freesurfer pipeline. It uses multiple T1s per imaging session, but does not combine them between sessions. Data is output to the dedicated FREESURFER directory, and should be exported to the MRI analysis folders using \texttt{fsexport.py}.

%%%%%%%%%%%%%%%%%%%%%%%%%%%%%%%%%%%%%%%%%%%%%%%%%%%%%%%%%%%%%%%%%%%%%%%%%%%%%%%
\subsubsection{fsexport.py}
Usage: \texttt{fsexport.py <data\_directory> <experiment>}

\begin{blockquote}
data\_directory -- full path to your MRI/WORKING directory. \\
experiment -- name of the experiment being analyzed. \
\end{blockquote}

\noindent Imports processed T1s from Freesurfer to the experiment directory.

%%%%%%%%%%%%%%%%%%%%%%%%%%%%%%%%%%%%%%%%%%%%%%%%%%%%%%%%%%%%%%%%%%%%%%%%%%%%%%%
%%%%%%%%%%%%%%%%%%%%%%%%%%%%%%%%%%%%%%%%%%%%%%%%%%%%%%%%%%%%%%%%%%%%%%%%%%%%%%%
\subsection{Pre-Processing}

This contains the lion's share of the pipeline. Every run of EPItome begins with \texttt{init\_EPI}, which contains a non-contentious set of pre-processing steps for EPI images. THe following stages can be chained together at will to preform de-noising, spatial transformations, projections to surface-space, and spatial smoothing.

%%%%%%%%%%%%%%%%%%%%%%%%%%%%%%%%%%%%%%%%%%%%%%%%%%%%%%%%%%%%%%%%%%%%%%%%%%%%%%%
\subsubsection{init\_EPI}
Usage: \texttt{init\_EPI <data\_quality> <del\_tr> <t\_pattern> <normalization> <masking>}

\begin{blockquote}
data\_quality -- `low' for poor internal contrast, otherwise `high'. \\
del\_tr -- number of TRs to remove from the beginning of the run. \\
t\_pattern -- slice-timing at acquisition (from AFNI's 3dTshift). \
normalization -- voxel wise time series normalization. One of `zscore', `pct', `demean'. \\
masking -- EPI brain masking tolerance. One of `loose', `normal' `tight'. \
\end{blockquote}

\noindent Works from the raw data in each RUN folder. It performs general pre-processing for all fMRI data:

\begin{itemize} \itemsep-2pt
    \item{Orients data to RAI}
    \item{Deletes initial time points (optionally)}
    \item{Removes data outliers}
    \item{Slice time correction}
    \item{Deobliques \& motion corrects data}
    \item{Creates session mean deskulled EPIs and whole-brain masks}
    \item{Scales and optionally normalizes each time series}
    \item{Calculates various statistics + time series}
\end{itemize}

Times series normalization can be accomplished in one of two ways: percent signal change, and scaling. For percent signal change, the data is normalized by the mean of each time series to mean = 100. A deviation of 1 from this mean value indicates 1\% signal change in the data. This is helpful for analyzing only relative fluctuations in the signal and is best at removing inter-session/subject/run variability, although it can also introduce rare artifacts in small localized regions of the images and may not play well with multivariate techniques such as partial least squares without accounting for these artifacts. Alternatively, one can scale the data, which applies single scaling factor to all voxels such that the global mean of the entire run = 1000. This will help normalize baseline shifts across sessions, runs, and participants. Your selection here might be motivated by personal preference, or in rarer cases, analytic requirements. When in doubt, it is safe to select `off', as scaling can be done later by hand, or `scale' if one is doing a simple GLM-style analysis. `pct' should be used by those with a good reason.

Masking options are provided to improve masking performance across various acquisition types, but it is very hard to devise a simple one-size fits all solution for this option. Therefore the QC outputs will be very important for ensuring good masking, and these options may need to be tweaked on a site-by-site basis. Luckily, many analysis methods do not rely heavily on mask accuracy. In cases that do, such as partial least squares / ICA / PCA analysis, close attention should be paid to the output of this step. Hopefully the `loose', `normal', and `tight' nomenclature are self-explanatory. Generally, it is best to start with normal, and adjust if required. 

%%%%%%%%%%%%%%%%%%%%%%%%%%%%%%%%%%%%%%%%%%%%%%%%%%%%%%%%%%%%%%%%%%%%%%%%%%%%%%%
\subsubsection{combine\_volumes}
Usage: \texttt{combine\_volumes <func1\_prefix> <func2\_prefix>}

\begin{blockquote}
func1\_prefix -- functional data prefix (eg., smooth in func\_smooth).
func2\_prefix -- functional data prefix (eg., smooth in func\_smooth).
\end{blockquote}

\noindent Combines two functional files via addition. Intended to combine the outputs of \texttt{surfsmooth} \& \texttt{surf2vol} with \texttt{volsmooth}, but could be used to combine other things as well. The functional files should not have non-zeroed regions that overlap, or the output won't make much sense.

%%%%%%%%%%%%%%%%%%%%%%%%%%%%%%%%%%%%%%%%%%%%%%%%%%%%%%%%%%%%%%%%%%%%%%%%%%%%%%%
\subsubsection{linreg\_calc\_AFNI}
Usage: \texttt{linreg\_calc\_AFNI <cost> <reg\_dof> <data\_quality>}

\begin{blockquote}
cost -- cost function minimized during registration. \\
reg\_dof -- `big\_move' or `giant\_move' (from align\_EPI\_anat.py). \\
data\_quality -- `low' for poor internal contrast, otherwise `high'. \
\end{blockquote}

\noindent Uses AFNI's align\_EPI\_anat.py to calculate linear registration between EPI <--> T1 <--> MNI152, and generate an EPI template registered to T1 \& T1 registered to EPI (sessionwise). Specific options can be found in the command-line interface's help function.

%%%%%%%%%%%%%%%%%%%%%%%%%%%%%%%%%%%%%%%%%%%%%%%%%%%%%%%%%%%%%%%%%%%%%%%%%%%%%%%
\subsubsection{linreg\_calc\_FSL}
Usage: \texttt{linreg\_calc\_FSL <cost> <reg\_dof> <data\_quality>}

\begin{blockquote}
cost -- cost function minimized during registration (see FSL FLIRT). \\
reg\_dof -- 6, 7, 9, or 12 degrees of freedom (see FSL FLIRT), \\
data\_quality -- `low' for poor internal contrast, otherwise `high'. \
\end{blockquote}

\noindent Uses FSL's FLIRT to calculate linear registration between EPI <--> T1 <--> MNI152, and generate an EPI template registered to T1 \& T1 registered to EPI (sessionwise). Specific options can be found in the command-line interface's help function.

%%%%%%%%%%%%%%%%%%%%%%%%%%%%%%%%%%%%%%%%%%%%%%%%%%%%%%%%%%%%%%%%%%%%%%%%%%%%%%%
\subsubsection{linreg\_EPI2MNI\_AFNI}
Usage: \texttt{linreg\_EPI2MNI\_AFNI <func\_prefix> <voxel\_dims>}

\begin{blockquote}
func\_prefix -- functional data prefix (eg.,smooth in func\_smooth). \\
voxel\_dims -- target voxel dimensions (isotropic). \
\end{blockquote}

\noindent Prepares data for analysis in MNI standard space.

%%%%%%%%%%%%%%%%%%%%%%%%%%%%%%%%%%%%%%%%%%%%%%%%%%%%%%%%%%%%%%%%%%%%%%%%%%%%%%%
\subsubsection{linreg\_EPI2MNI\_FSL}
Usage: \texttt{linreg\_EPI2MNI\_FSL <func\_prefix> <voxel\_dims>}

\begin{blockquote}
func\_prefix -- functional data prefix (eg., smooth in func\_smooth). \\
voxel\_dims -- target voxel dimensions (isotropic). \
\end{blockquote}

\noindent Prepares data for analysis in MNI standard space.

%%%%%%%%%%%%%%%%%%%%%%%%%%%%%%%%%%%%%%%%%%%%%%%%%%%%%%%%%%%%%%%%%%%%%%%%%%%%%%%
\subsubsection{linreg\_FS2EPI\_AFNI}
Usage: \texttt{linreg\_FS2EPI\_AFNI}

\noindent Brings Freesurfer atlases in register with single-subject EPIs.

%%%%%%%%%%%%%%%%%%%%%%%%%%%%%%%%%%%%%%%%%%%%%%%%%%%%%%%%%%%%%%%%%%%%%%%%%%%%%%%
\subsubsection{linreg\_FS2EPI\_FSL}
Usage: \texttt{linreg\_FS2EPI\_FSL}

\noindent Brings Freesurfer atlases in register with single-subject EPIs.

%%%%%%%%%%%%%%%%%%%%%%%%%%%%%%%%%%%%%%%%%%%%%%%%%%%%%%%%%%%%%%%%%%%%%%%%%%%%%%%
\subsubsection{linreg\_FS2MNI\_FSL}
Usage: \texttt{linreg\_FS2MNI\_FSL}

\noindent Brings Freesurfer atlases in register with MNI standard space.

%%%%%%%%%%%%%%%%%%%%%%%%%%%%%%%%%%%%%%%%%%%%%%%%%%%%%%%%%%%%%%%%%%%%%%%%%%%%%%%
\subsubsection{gen\_regressors}
Usage: \texttt{gen\_regressors <func\_prefix>}

\begin{blockquote}
func\_prefix -- functional data prefix (eg.,smooth in func\_smooth).
\end{blockquote}

\noindent Creates a series of regressors from fMRI data and a freesurfer segmentation: 

\begin{itemize} \itemsep-2pt
	\item{white matter + eroded mask}
	\item{ventricles + eroded mask}
	\item{grey matter mask}
	\item{brain stem mask}
	\item{dialated whole-brain mask}
	\item{draining vessels mask}
	\item{local white matter regressors + 1 temporal lag}
	\item{ventricle regressors + 1 temporal lag}
	\item{draining vessel regressors + 1 temporal lag}
\end{itemize}

%%%%%%%%%%%%%%%%%%%%%%%%%%%%%%%%%%%%%%%%%%%%%%%%%%%%%%%%%%%%%%%%%%%%%%%%%%%%%%%
\subsubsection{gen\_gcor}
Usage: \texttt{gen\_gcor <func\_prefix>}

\begin{blockquote}
func\_prefix -- functional data prefix (eg.,smooth in func\_smooth).
\end{blockquote}

\noindent Calls an AFNI script to calculate the global correlation for each concatenated set of runs (across all sessions). Useful for resting state functional connectivity experiments.

%%%%%%%%%%%%%%%%%%%%%%%%%%%%%%%%%%%%%%%%%%%%%%%%%%%%%%%%%%%%%%%%%%%%%%%%%%%%%%%
\subsubsection{filter}
Usage: \texttt{filter <func\_prefix> <det> <gs> <vent> <dv> <wm\_loc> <wm\_glo>}

\begin{blockquote}
func\_prefix -- functional data prefix (eg.,smooth in func\_smooth). \\
det -- polynomial order to detrend each voxel against. \\
gs -- if == on, regress mean global signal from each voxel. \\
vent -- if == on, regress mean ventricle signal from each voxel. \\
dv -- if == on, regress mean draining vessel signal from each voxel. \\
wm\_loc -- if == on, regress local white matter from target voxels. \\
wm\_glo -- if == on, regress global white matter for all voxels. \

\end{blockquote}

\noindent This computes detrended nuisance time series, fits each run with a computed noise model, and subtracts the fit. Computes temporal SNR. This program always regresses the motion parameters \& their first lags, as well as physiological noise regressors generated my McRetroTS if they are available. The rest are optional, and generally advisable save global mean regression.

%%%%%%%%%%%%%%%%%%%%%%%%%%%%%%%%%%%%%%%%%%%%%%%%%%%%%%%%%%%%%%%%%%%%%%%%%%%%%%%
\subsubsection{TRdrop}
Usage: \texttt{TRdrop <func\_prefix> <head\_size> <FD\_thresh> <DV\_thresh>}

\begin{blockquote}
func\_prefix -- functional data prefix (eg.,smooth in func\_smooth). \\
head\_size -- head radius in mm (def. 50 mm). \\
thresh\_FD -- censor TRs with $\Delta$ motion > $x$ mm (def. 0.3 mm). \\
thresh\_DV -- censor TRs with $\Delta$ GS change > $x$ \% (def. \href{http://upload.wikimedia.org/wikipedia/en/1/16/Drevil_million_dollars.jpg}{1000000} \%). \
\end{blockquote}

\noindent This removes motion-corrupted TRs from fMRI scans and outputs shortened versions for connectivity analysis (mostly). By default, DVARS regression is set of OFF by using a very, very high threshold.

%%%%%%%%%%%%%%%%%%%%%%%%%%%%%%%%%%%%%%%%%%%%%%%%%%%%%%%%%%%%%%%%%%%%%%%%%%%%%%%
\subsubsection{lowpass}
Usage: \texttt{lowpass <func\_prefix> <mask\_prefix> <filter> <cutoff>}

\begin{blockquote}
func\_prefix -- functional data prefix (eg.,smooth in func\_smooth). \\
mask\_prefix -- mask data prefix (eg., EPI\_mask in anat\_EPI\_mask). \\
filter -- filter type: `median', `average', `kaiser', or `butterworth'. \\
cutoff -- filter cuttoff: either window length, or cutoff frequency. \
\end{blockquote}

\noindent This low-passes input data using the specified filter type and cutoff. 

Both `median' and `average' filters operate in the time domain and therefore, the best cutoff values are odd (and must be larger than 1 to do anything). Time-domain filters are very good at removing high-frequency noise from the data without introducing any phase-shifts or ringing into the time series. When in doubt, a moving average filter with window length of 3 is a decent and conservative choice.

Alternatively, the `kaiser' and `butterworth' filters work in the frequency domain and accept a cutoff in Hz (people tend to use a default of 0.1). Both are implemented as bi-directional FIR filters. The kaiser window is high order and permits reasonably sharp rolloff with minimal passband ringing for shorter fMRI time series. The butterworth filter is of low order and achieves minimal passband ringing at the expense of passband roll off (in layman's terms, butterworth filters will retain more high-frequency content than a kaiser filter with equivalent cutoff). The effect of the passband ringing is an empirical question that would be best tested by the User.

%%%%%%%%%%%%%%%%%%%%%%%%%%%%%%%%%%%%%%%%%%%%%%%%%%%%%%%%%%%%%%%%%%%%%%%%%%%%%%%
\subsubsection{surfsmooth}
Usage: \texttt{surfsmooth <func\_prefix> <FWHM>}

\begin{blockquote}
func\_prefix -- functional data prefix (eg.,smooth in func\_smooth). \\
FWHM -- full-width half-maximum of the gaussian kernel convolved with the surface data. \
\end{blockquote}

\noindent This spatially-smooths cortical data along the surface mesh, estimated by Freesurfer.

%%%%%%%%%%%%%%%%%%%%%%%%%%%%%%%%%%%%%%%%%%%%%%%%%%%%%%%%%%%%%%%%%%%%%%%%%%%%%%%
\subsubsection{surf2vol}
Usage: \texttt{surf2vol <func\_prefix> <target\_prefix>}

\begin{blockquote}
func\_prefix -- functional data prefix (eg.,smooth in func\_smooth). \\
target\_prefix -- target data prefix (eg.,smooth in func\_smooth). \
\end{blockquote}

\noindent This projects surface data back into a functional volume with the same properties as <target\_prefix>.

%%%%%%%%%%%%%%%%%%%%%%%%%%%%%%%%%%%%%%%%%%%%%%%%%%%%%%%%%%%%%%%%%%%%%%%%%%%%%%%
\subsubsection{vol2surf}
Usage: \texttt{vol2surf <func\_prefix>}

\begin{blockquote}
func\_prefix -- functional data prefix (eg.,smooth in func\_smooth).
\end{blockquote}

\noindent Projects functional data from volume space to a Freesurfer generated cortical mesh.

%%%%%%%%%%%%%%%%%%%%%%%%%%%%%%%%%%%%%%%%%%%%%%%%%%%%%%%%%%%%%%%%%%%%%%%%%%%%%%%
\subsubsection{volsmooth}
Usage: \texttt{volsmooth <func\_prefix> <mask\_prefix> <FWHM>}

\begin{blockquote}
func\_prefix -- functional data prefix (eg., smooth in func\_smooth).
mask\_prefix -- mask data prefix (eg., EPI\_mask in anat\_EPI\_mask).
func\_prefix -- functional data prefix (eg.,smooth in func\_smooth).
\end{blockquote}

\noindent Re-samples a mask containing one or more labels to the functional data and smooths within unique values. All zero values in the mask are zeroed out in the output. The output of this can be combined with the outputs of \texttt{surfsmooth} \& \texttt{surf2vol} using \texttt{combine\_volumes}.

%%%%%%%%%%%%%%%%%%%%%%%%%%%%%%%%%%%%%%%%%%%%%%%%%%%%%%%%%%%%%%%%%%%%%%%%%%%%%%%
%%%%%%%%%%%%%%%%%%%%%%%%%%%%%%%%%%%%%%%%%%%%%%%%%%%%%%%%%%%%%%%%%%%%%%%%%%%%%%%
\subsection{Quality Control}

%%%%%%%%%%%%%%%%%%%%%%%%%%%%%%%%%%%%%%%%%%%%%%%%%%%%%%%%%%%%%%%%%%%%%%%%%%%%%%%
% References, both on server building and MRI analysis
%%%%%%%%%%%%%%%%%%%%%%%%%%%%%%%%%%%%%%%%%%%%%%%%%%%%%%%%%%%%%%%%%%%%%%%%%%%%%%%
\section{Further Reading}

%% End of document. Go to reference list.
\newpage 
\bibliographystyle{apalike}
\bibliography{harford}
\end{document}
